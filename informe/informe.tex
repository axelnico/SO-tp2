% ******************************************************** %
%              TEMPLATE DE INFORME ORGA2 v0.1              %
% ******************************************************** %
% ******************************************************** %
%                                                          %
% ALGUNOS PAQUETES REQUERIDOS (EN UBUNTU):                 %
% ========================================
%                                                          %
% texlive-latex-base                                       %
% texlive-latex-recommended                                %
% texlive-fonts-recommended                                %
% texlive-latex-extra?                                     %
% texlive-lang-spanish (en ubuntu 13.10)                   %
% ******************************************************** %


\documentclass[a4paper]{article}
\usepackage[spanish]{babel}
\usepackage[utf8]{inputenc}
\usepackage{charter}   % tipografia
\usepackage{graphicx}
%\usepackage{makeidx}
\usepackage{paralist} %itemize inline

%\usepackage{float}
%\usepackage{amsmath, amsthm, amssymb}
%\usepackage{amsfonts}
%\usepackage{sectsty}
%\usepackage{charter}
%\usepackage{wrapfig}
\usepackage{listings}
\lstset{language=C}
\usepackage{caption}
\usepackage{color}

% \setcounter{secnumdepth}{2}
\usepackage{underscore}
\usepackage{caratula}
\usepackage{url}
\usepackage[document]{ragged2e}
\usepackage[export]{adjustbox}
\usepackage{subcaption}
\usepackage{floatrow}


% ********************************************************* %
% ~~~~~~~~              Code snippets             ~~~~~~~~~ %
% ********************************************************* %

\usepackage{color} % para snipets de codigo coloreados
\usepackage{fancybox}  % para el sbox de los snipets de codigo

\definecolor{litegrey}{gray}{0.94}

\newenvironment{codesnippet}{%
	\begin{Sbox}\begin{minipage}{\textwidth}\sffamily\small}%
	{\end{minipage}\end{Sbox}%
		\begin{center}%
		\vspace{-0.4cm}\colorbox{litegrey}{\TheSbox}\end{center}\vspace{0.3cm}}



% ********************************************************* %
% ~~~~~~~~         Formato de las páginas         ~~~~~~~~~ %
% ********************************************************* %

\usepackage{fancyhdr}
\usepackage{parskip}
\pagestyle{fancy}

%\renewcommand{\chaptermark}[1]{\markboth{#1}{}}
\renewcommand{\sectionmark}[1]{\markright{\thesection\ - #1}}

\fancyhf{}

\fancyhead[LO]{Sección \rightmark} % \thesection\
\fancyfoot[LO]{\small{Nicolas Bukovits, Kevin Frachtenberg Goldsmit, Laura Muiño}}
\fancyfoot[RO]{\thepage}
\renewcommand{\headrulewidth}{0.5pt}
\renewcommand{\footrulewidth}{0.5pt}
\setlength{\hoffset}{-0.8in}
\setlength{\textwidth}{16cm}
%\setlength{\hoffset}{-1.1cm}
%\setlength{\textwidth}{16cm}
\setlength{\headsep}{0.5cm}
\setlength{\textheight}{25cm}
\setlength{\voffset}{-0.7in}
\setlength{\headwidth}{\textwidth}
\setlength{\headheight}{13.1pt}
\setlength{\parindent}{4em}
\setlength{\parskip}{\baselineskip}

\renewcommand{\baselinestretch}{1.1}  % line spacing

% ******************************************************** %


\begin{document}


\thispagestyle{empty}
\materia{Sistemas Operativos}
\submateria{Primer Cuatrimestre de 2018}
\titulo{Trabajo Práctico 2}
\subtitulo{Sistemas distribuidos}
\integrante{Nicolas Bukovits}{546/14}{nicko_buk@hotmail.com}
\integrante{Kevin Frachtenberg}{247/14}{kevinfra94@gmail.com}
\integrante{Laura Muiño}{399/11}{lauramuino2@gmail.com}
\maketitle

%{\small\textbf{\flushleft{Resumen}}\\
\abstract {En el siguiente trabajo pr\'actico, se realiz\'o una implementaci\'on de un protocolo de envío de mensajes entre procesos concurrentes. En el mismo se desarrolló un sistema de blockchain.

%\newpage

%\thispagestyle{empty}
%\vfill


\thispagestyle{empty}
\vspace{3cm}
\tableofcontents
\newpage


%\normalsize
\newpage



\section{Analisis del protocolo}
\subsection{¿Puede este protocolo producir dos o más blockchains que nunca converjan?}

EL protocolo puede producir dos o más blockchains que nunca converjan. Si se tienen dos nodos A y B que cuando B comunica a A que encontró un bloque nuevo, y si A, antes de recibir el mensaje de B, encontró un bloque nuevo con el mismo índice que el nuevo bloque de B, entonces cuando A reciba el mesaje de B, va a descartar su bloque porque va a apostar por su propia cadena. Este esquema se puede repetir una cantidad de veces indeterminadas generan dos bifurcaciones que nunca converjen.


\subsection{¿Cómo afecta la demora o la pérdida en la entrega de paquetes al protocolo?}

La demora en la entrega de paquetes puede producir dos bifuraciones como lo mencionado en la pregunta anterior. La pérdida de mensajes puede producir que un nodo quede lo suficientemente desactualizado como para que el indice de su último nodo esté a una distancia mayor que la máxima diferencia tolerada de bloques generando nuevamente una bifurcación que podría no converjer.  

\subsection{¿Cómo afecta el aumento o la disminución de la dificultad del Proof-of-Work a los conflictos entre nodos y a la convergencia?}


Se agregó el archivo test-1.cpp el cual incluye tests particulares sobre ConcurrentHashMap. El archivo tiene tres partes en el cual se prueban las funciones member,addAndInc y maximum del ConcurrentHashMap, las cuales para poder ejecutarse prueban también el constructor del mismo. Para poder realizar los tests fue necesario agregar funciones auxiliares que se usan para este fin las cuales son add, Inc y count_word. Estas funciones fueron agregadas para testear member,addAndInc y maximum sin asumir que las demas funcionan excepto el maximum. Los tests se desarrollaron de manera incremental empezando con member, luego una vez que se testeo correctamente member, se testeó addAndInc y finalmente maximum.


\end{document}
